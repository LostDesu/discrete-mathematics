\documentclass[a4paper,12pt]{article} 
\usepackage[T2A]{fontenc}			
\usepackage[utf8]{inputenc}			
\usepackage[english,russian]{babel}	
\usepackage{amsmath,amsfonts,amssymb,amsthm,mathtools} 
\usepackage{wasysym}
\usepackage{amsmath}
\everymath{\displaystyle}

\author{конспект от TheLostDesu}
\title{Математическая индукция}
\date{\today}


\begin{document}
\maketitle
\section{Индукция}
Рассмотрим множество натуральных чисел. Договоримся, что 0 - также натуральное число.
На нем можно использовать мат.индукцию. \\
Пусть $\phi (x)$ - предикат на $\mathbb{N}$. Тогда $(\phi (0) \wedge \forall n (\phi(n) \rightarrow \phi (n + 1) \forall n \phi(n)$. \\
\\
Как интуитивно доказать, например, что мат. индукция работает для натуральных чисел?
Рассмотрим, например $\phi(5)$. Для любого $\phi(n) \rightarrow \phi (n + 1)$. Тогда, так как верно $\phi (0)$, то верно $\phi (1)$. Из верности $\phi (1)$ следует верность $\phi (2)$. И так далее. Тогда, можно заметить, что для любого $n \in \mathbb{N}$ верно $\phi (n)$.\\
Пример: Пусть все кванторы по натуральным числам. Доказать, что $\forall n (n \geq 3 \rightarrow \exists a_1, a_2...a_n > 0, 1 = \frac{1}{a_1} + \frac{1}{a_2} + ... + \frac{1}{a_n}$ при этом все $a_i$ попарно различны. \\
Для нуля получается правда(посылка ложна -> вся импликация истинна). \\
Теперь надо рассмотреть случаи. Если $n = 0$, $1$, то для $n + 1$ высказывание верно\footnote{См случай для нуля}. Если $n = 2$, то для $n + 1$ можно подобрать пример, например $\frac{1}{2} + \frac{1}{3} + \frac{1}{6}$. Пусть верно для $n$. Докажем для $n + 1$. Рассмотрим сумму для $n$. Она имеет вид $\frac{1}{a_1} + \frac{1}{a_2} + ... + \frac{1}{a_n}$. Рассмотрим сумму $\frac{1}{2a_1} + \frac{1}{2a_2} +...+ \frac{1}{2a_n}$. Она равна $\frac{1}{2}$. Также стоит отметить, что все числа различны(если $a_1 \neq a_2$, то и $2a_1 \neq 2a_2$). Алсо, ни одно из чисел не равно $/frac{1}{2}$ по построению этих чисел. Просто добавим к нему $\frac{1}{2}$, сумма станет равна 1. А значит, изначальное высказывание верно.

\section{Принципы}
Идея заключается в том, что для вывода $\phi (n + 1)$ разрешить использовать верность $\phi (n), \phi (n - 1), ... , phi(0)$\\
$\forall n(\forall m < n \phi (m) \rightarrow \phi(n)) \rightarrow \forall n \phi(n)$. prog(n)\footnote{Прогрессивность n}

Принцип наименьшего числа. Если свойству $\phi $ удвлетворяет какое-то число, значит есть наименьшее число со свойством $\phi $. 
\end{document}