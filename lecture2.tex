\documentclass[a4paper,12pt]{article} 
\usepackage[T2A]{fontenc}			
\usepackage[utf8]{inputenc}			
\usepackage[english,russian]{babel}	
\usepackage{amsmath,amsfonts,amssymb,amsthm,mathtools} 
\usepackage{wasysym}
\usepackage{amsmath}

\author{конспект от TheLostDesu}
\title{Логика. Продолжение}
\date{\today}


\begin{document}
\maketitle
\section{Зачем нужна логика}
Логика нужна, чтобы не ошибатся. Она позволяет из истиных посылок получать истиные рассуждения. Пусть мы представили какое-то рассуждение в виде логического выражения. Тогда если это, например, тавтология, то рассуждение всегда верно, о чем бы мы не рассуждали. Если рассуждение корректно, то при истинности всех посылок оно должно быть истинно.

Пример. Сегодня четверг. Если сегодня четверг, то идет дождь. Следовательно идет дождь.\\
Назовем атом "Сегодня четверг" - $A$. "Идет дождь" - $B$. \\
Тогда формально это будет выглядеть, как $(A \wedge (A\rightarrow B)) \rightarrow B$.
Следует заметить, что это - тавтология. Значит, что рассуждение корректно, ведь если истинны посылки, истинно заключение\footnote{В данном случае заключение истинно всегда}.\\
Пример корректного рассуждения, которое не является тавтологией. Если натуральное $n$ делится на 4, то $n$ делится на 2. Это корректное рассуждение в силу свойств натуральных числел, но тавтологией оно не является. Более того, причина корректности этого рассуждения не улавливается логическими связками.\\
\section{Предикат}
Предикат с $n$ переменными $x_1...x_n$ - выражение, которое превращается в высказывание, если вместо переменных $x_1...x_n$ подставить имена подходящих объектов. Можно считать, что высказывание - предикат с нулем праметров. Обычно предикаты с одним параметром называются унарными. С двумя - бинарными. С тремя - тернарными.
По факту предикат - высказывание с параметрами.\\
Пусть есть высказывание <<3 - четное число>>. Это высказывание ложно. А вот <<x - четное число>> не всказывание, так как у него нет определенного истинного значения. Но если подставить вместо x какое-то число, то это станет высказыванием. <<x - четное число>> называется предикатом.

В предикат нельзя подставлять все что угодно. Например <<2 мыла 3>>, или <<мама = рама + 7>> высказывания не имеющие смысла. Поэтому в каждый предикат можно подставлять что-то из определенной области. 

Предикаты, как и высказывания можно объединять логическими связками. 

\section{Кванторы}
Возьмем предикат.  <<$x$ четно>>. \\
Возьмем высказывания <<Существует $x$, такой что $x$ четно>>. <<Существует ровно $1$ $x$, такой что $x$ четно>>. <<Большинство $x$ таковы, что $x$ четно>>. <<Существует ровно $1$ $x$, такой что $x$ четно>>. <<Существует бесконечное число $x$, таких что $x$ четно.>>. Все эти высказывания похожи тем, что вместо $x$ нет смысла что - то подставлять. Эти выражения можно сокращать, используя кванторы. Пусть есть предикат $\phi$. тогда

Квантор всеобщности\footnote{в LaTex forall}. Обозначается $\forall$. $\forall x \phi(x, y_1, y_2...y_n)$. Для всех $x$ $\phi(x, y_1, y_2...y_n0$. 

Квантор существования \footnote{В LaTex exists}. Обозначается $\exists$. $\exists x \phi(x, y_1, y_2...y_n)$. Существует $x$, такой что $\phi(x, y_1, y_2...y_n)$.

Существует и единственно. Обозначается $\exists !x$. Однако, его можно не вводить, так как его можно вывести из двух <<основных>> кванторов.

Истинность квантора зависит от области x.
\section{vacous truth}
<<Текущий король Франции лыс>>\footnote{Во Франции сейчас нет короля}. Если бы это высказывание утверждало существование короля Франции, то оно очевидно было бы ошибочно. Но, на языке математики это выглядит, как $\forall x($если $x$ - нынешний король Франции $\rightarrow x лыc$. x лежит в множестве всех людей. $x$ - нынешний король Франции никогда не станет правдой. Тогда высказывание всегда будет правдой. Тогда это высказывание истинно. Аналогично, <<Текущий король франции волосатый>> тоже верное высказывание. 

Это чинится ограниченным квантором существования.
Пусть A - подмножество области. \\
$\forall x \in A \phi (x)$ $\Leftrightarrow$ $\forall x(x \in A \rightarrow \phi (x)$\\
и\\
$\exists x \in A \phi(x)$ $\Leftrightarrow \exists x(x \in A \wedge \phi (x)$

\end{document}