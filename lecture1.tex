\documentclass[a4paper,12pt]{article} 
\usepackage[T2A]{fontenc}			
\usepackage[utf8]{inputenc}			
\usepackage[english,russian]{babel}	
\usepackage{amsmath,amsfonts,amssymb,amsthm,mathtools} 
\usepackage{wasysym}
\usepackage{amsmath}

\author{конспект от TheLostDesu}
\title{Логика}
\date{\today}


\begin{document}
\maketitle
\section{Основы логики}
Существуют повествователные предложения. Они бывают истиными и ложными\\
Так, например, <<2=3>> - ложь, а <<Сегодня суббота>> - правда\footnote{лекция проходит в субботу}. Однако, это не всегда суббота, поэтому это высказывание не всегда справедливо, и зависит от некоторого <<сегодня>>\\
<<Существует нечетное совершенное число>> - сейчас нет доказательства того, что их не существует, но и нет найденных совершенных чисел - поэтому об этом высказывании говорить нет смысла.
Рассмотрим еще один пример. <<Это предложение истинно>> - Это высказывание может быть и истинным, так и ложным, так как оно описывает действительность в самом предложении.\\
Также, есть и пародоксальный пример <<Это предложение ложно>> - не может быть истинно и не может быть ложно\footnote{Читатель может проверить, что происходит, когда это истино или ложно}. 

Однако, в математике используются <<высказывания>> - это некотарая модель повествовательного предложения, которое всегда либо истинно, либо ложно, но никогда не истинно и ложно одновременно.\\
\section{Составные высказывания}
<<2=3 или 7=5>> - составное высказывание. Его можно разбить на 2 более простых высказывания.\\
Есть множество способов построить составное высказывание. Разобъем их на 2 группы:\\
I) A или B; A и B; если A, то B; либо A, либо B; не A; A равносильно B...\\
II) Мне нравится, что A, Все студенты знают, что A\footnote{Далее ВСЗЧA}

В первой группе результат высказывания можно однозначно определить, зная справедливость простых высказываний. Истиность II группы высказываний сложно зависит от A, а не только от его истинности. Высказывания из II группы не будут рассматриватся.
\subsection{Логичесие связки}
Это способ образования новых высказываний, такой что истинность целого полностью определяется истинностью его частей.\\

\begin{tabular}{||l | l | l | l | l | l||}
\hline
A & B & A и B & не A & A или B & если A, то B\\
\hline
\hline
0 & 0 & 	0 & 1 & 0 & 1\\
\hline
0 & 1 & 0 & 1 & 1 & 1\\
\hline
1 & 0 & 0 & 0 & 1 & 0\\
\hline
1 & 1 & 1 & 0 & 1 & 1\\
\hline
\end{tabular}

В математике часто используют разные символические сокращения. \\
И: $\wedge$ - коньюнкция\footnote{Для латеха wedge}\\
Или: $\vee$ - дизъюнкция\footnote{Для латеха - vee}\\
Отрицание: $\urcorner$\footnote{Для латеха - urcorner}\\
Следование: $\leftarrow$ - импликация\footnote{Для латеха - leftarrow}\\
Равносильность: $\Leftrightarrow$\footnote{Для латеха - Leftrightarrow}

Левая часть импликации называют посылкой; правую - заключением.

Стоит отметить, что когда высказывание A истинно - следует писать [A] = 1, а не A = 1. Это позволяет убрать недопонимание во время приравнивания высказываний.\footnote{Например, если записать, что Великая Теорема Ферма истина, как <<ВТФ>> = 1, и <<0=0>> = 1, то можно будет сказать, что 0=0 = ВТФ, что не совсем верно.}

\subsection{Тавтология}
Назовем F($A_1$, $A_2$, $A_3$,...,$A_n$) - высказывание зависящее от $A_1$, $A_2$, $A_3$, ..., $A_n$. 

Тавтология - Если F($A_1$, $A_2$, $A_3$,..., $A_n$) истинно при любых $A_1$, $A_2$, $A_3$,...,$A_n$. 

Есть несколько способов проверки на тавтологию:\\
1. Перебрать все возможные варианты входных данных и построить таблицу истинности.\\
2. Подставить в значение 0, и попробовать подобрать значения входных данных, для которых 0 - значение.\\
3. Построить таблицу. В правую часть записывать то, что должно быть истинным, в левую - то, что должно быть ложным, для того, чтобы значение стало 0. Тогда, если в обоих столбцах записано одно значение, то тогда высказывание - тавтология.

\subsection{Эквивалентность}
Некоторые высказывания приобретают одинаковые значения, их называют эквивалентными. Этим можно пользоваться при упрощении выражений по некоторым формулам.\\
1) $A \vee A \Leftrightarrow A\wedge A \Leftrightarrow A$\\
2) $A \vee 1 \Leftrightarrow 1$
3) $A \wedge 1 \Leftrightarrow A$
4) $A \vee \urcorner A \Leftrightarrow 1$\\
5) $A \wedge \urcorner A \Leftrightarrow 0$\\
6) $(A \vee B) \vee C \Leftrightarrow A\vee(B \vee C)$\\
7) $A\rightarrow B \Leftrightarrow \urcorner A\vee B$\\
8) $A\rightarrow B \Leftrightarrow \urcorner B \rightarrow \urcorner A$\\
9) $A\rightarrow (B \rightarrow C) \Leftrightarrow (A\wedge B) \rightarrow C$\\
10) $\urcorner(A \wedge B) \Leftrightarrow \urcorner A \vee \urcorner B$\\
11) $\urcorner(A \vee B) \Leftrightarrow \urcorner A \wedge \urcorner B$\\
Этим способом можно находить тавтологии. Просто следует упростить выражение до его минимальной длины. Тогда станет возможно легко проверить это даже глазами.
\end{document}